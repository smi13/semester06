\section{Определение и свойства автоморфизма Фробениуса}

Пусть $ \BB{K} $ - поле, char$(\BB{K}) = p $,  $ \galf \subset \BB{K} $.

\begin{defn}
  $ \frob : \BB{K} \rightarrow \BB{K}, ~ \frob(x) = x^p $ назовем \emph{автоморфизмом Фробениуса}. 
\end{defn}

$ \BB{K} $ - векторное пространство над $ \galf $.

\begin{thm}
  $ \frob $ - линейный оператор в $ \BB{K} $ над $ \galf $.
\end{thm}

\begin{proof}
     \[ \frob(xy) = x^py^p = \frob(x)\frob(y) \]
     \[ \frob(x + y) = (x + y)^p = \underset{i = 0}{\overset{p}\sum}C_{p}^{i}x^{i}y^{p-i} \] 
     \[ \forall i \in [1,\dots,p-1] ~~  C_{p}^{i} ~ \vdots ~ p \]  
     \[ (x+y)^p = x^p + y^p = \frob(x) + \frob(y) \] \newline
     \[ \frob(ax + by) = a^px^p + b^py^p \]
     \[ a, b \in \galf \Rightarrow a^p = a, b^p = b \]
     \[ \frob(ax + by) = ax^p + by^p \]
\end{proof}

Также можно рассматривать $ \phi : \BB{K}[x] \rightarrow \BB{K}[x] $, в таком случае автоморфизм Фробениуса 
будет линейным оператором в $ \BB{K}[x]/(f) $ над $ \galf $, если $ f $ - неприводимый многочлен. 

\begin{defn}
  \emph{Алгеброй} называют векторное пространство или модуль с умножением.
\end{defn}

Пример:

\begin{enumerate}
  \item $ \BB{R}^3 $
  \item $ \BB{R} $ - $\BB{Z}$-алгебра
  \item $ A, B $ - кольца. $ A \subset B $. $ B $ - $A$-алгебра
\end{enumerate}

